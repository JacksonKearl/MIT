%%%%%%%%%%%%%%%%%%%%%%%%%%%%%%%%%%%%%%%%%%%%%%%%%%%
%% LaTeX book template                           %%
%% Author:  Amber Jain (http://amberj.devio.us/) %%
%% License: ISC license                          %%
%%%%%%%%%%%%%%%%%%%%%%%%%%%%%%%%%%%%%%%%%%%%%%%%%%%

\documentclass[a4paper,11pt]{book}
\usepackage[T1]{fontenc}
\usepackage[utf8]{inputenc}
\usepackage{lmodern}
\usepackage{csquotes}
\usepackage{amsmath}




\usepackage[margin=1.25in]{geometry}

%%%%%%%%%%%%%%%%%%%%%%%%%%%%%%%%%%%%%%%%%%%%%%%%%%%%%%%%%
% Source: http://en.wikibooks.org/wiki/LaTeX/Hyperlinks %
%%%%%%%%%%%%%%%%%%%%%%%%%%%%%%%%%%%%%%%%%%%%%%%%%%%%%%%%%
\usepackage{hyperref}
\usepackage{graphicx}
\usepackage[english]{babel}

%%%%%%%%%%%%%%%%%%%%%%%%%%%%%%%%%%%%%%%%%%%%%%%%%%%%%%%%%%%%%%%%%%%%%%%%%%%%%%%%
% 'dedication' environment: To add a dedication paragraph at the start of book %
% Source: http://www.tug.org/pipermail/texhax/2010-June/015184.html            %
%%%%%%%%%%%%%%%%%%%%%%%%%%%%%%%%%%%%%%%%%%%%%%%%%%%%%%%%%%%%%%%%%%%%%%%%%%%%%%%%
\newenvironment{dedication}
{
   \cleardoublepage
   \thispagestyle{empty}
   \vspace*{\stretch{1}}
   \hfill\begin{minipage}[t]{0.66\textwidth}
   \raggedright
}
{
   \end{minipage}
   \vspace*{\stretch{3}}
   \clearpage
}

%%%%%%%%%%%%%%%%%%%%%%%%%%%%%%%%%%%%%%%%%%%%%%%%
% Chapter quote at the start of chapter        %
% Source: http://tex.stackexchange.com/a/53380 %
%%%%%%%%%%%%%%%%%%%%%%%%%%%%%%%%%%%%%%%%%%%%%%%%
\makeatletter
\renewcommand{\@chapapp}{}% Not necessary...
\newenvironment{chapquote}[2][2em]
  {\setlength{\@tempdima}{#1}%
   \def\chapquote@author{#2}%
   \parshape 1 \@tempdima \dimexpr\textwidth-2\@tempdima\relax%
   \itshape}
  {\par\normalfont\hfill--\ \chapquote@author\hspace*{\@tempdima}\par\bigskip}
\makeatother

%%%%%%%%%%%%%%%%%%%%%%%%%%%%%%%%%%%%%%%%%%%%%%%%%%%
% First page of book which contains 'stuff' like: %
%  - Book title, subtitle                         %
%  - Book author name                             %
%%%%%%%%%%%%%%%%%%%%%%%%%%%%%%%%%%%%%%%%%%%%%%%%%%%

% Book's title and subtitle
\title{\Huge \textbf{18.02: Multivariable Calculus} \\ \huge Massachusetts Institute of Technology \\ \large As taught by \textsc{Prof. John Bush}}
% Author
\author{\textsc{Jackson Kearl}}


\begin{document}

\maketitle

%%%%%%%%%%%%%%%%%%%%%%%%%%%%%%%%%%%%%%%%%%%%%%%%%%%%%%%%%%%%%%%%%%%%%%%%
% Auto-generated table of contents, list of figures and list of tables %
%%%%%%%%%%%%%%%%%%%%%%%%%%%%%%%%%%%%%%%%%%%%%%%%%%%%%%%%%%%%%%%%%%%%%%%%
\let\cleardoublepage\clearpage
\tableofcontents
\let\cleardoublepage\clearpage
\mainmatter

%%%%%%%%%%%
% Preface %
%%%%%%%%%%%
\chapter*{Introduction}

\begin{chapquote}{Lewis Carroll, \textit{Alice in Wonderland}}
``Begin at the beginning,'' the King said, gravely, ``and go on till you
come to an end; then stop.''
\end{chapquote}

\section*{Course Content}
This course covers aspects of calculus involving:
\begin{displayquote}
Calculus of several variables. Vector algebra in 3-space, determinants, matrices.
 Vector-valued functions of one variable, space motion. Scalar functions of several variables: partial differentiation, gradient, optimization techniques.  Double integrals and line integrals in the plane; exact differentials and  conservative fields; Green's theorem and applications, triple integrals, line and surface integrals in space, Divergence theorem, Stokes' theorem;  applications.\footnote{\url{http://student.mit.edu/catalog/m18a.html##18.02}}
\end{displayquote}


\section*{Structure of Notes}
% You might want to add short description about each chapter in this book.
Each chapter will focus on some broad concept of the course (along the lines of one chapter per test). Within a chapter, each section approximately corresponds to one lecture. Each section contains an overview of the lecture's concepts, followed by example problems taken from the lecture, recitations, or from homework.

\section*{Distribution}
All relevant notes and information are made available on github\footnote{\url{https://github.com/JacksonKearl/MIT}}.
This includes:
\begin{itemize}
  \item Plaintext notes from lecture and recitation
  \item The raw .tex source for these PDF's
  \item These PDF files
\end{itemize}

\noindent Licensed under Creative Commons Attribution-ShareAlike 4.0 International License.\footnote{\url{http://creativecommons.org/licenses/by-sa/4.0/}}


%%%%%%%%%%%%%%%%
% NEW CHAPTER! %
%%%%%%%%%%%%%%%%
\chapter{Matrices, Vectors, and their Basic Operations}

\begin{chapquote}{Professor J. W. Bush}
17 39
\end{chapquote}

\section{Vectors}

\subsection{Scalars vs. Vectors}
Before entering into vectors, it is necessary to define the concept of a scalar. Most every value dealt with in day to day life is a scalar: from weights of an apple, to the time it takes to fall on your head. \\
However, things that also have a direction, such as traveling 25 km \emph{northwest} to get to Welesley, are vectors. \\
Put simply: a vector is a scalar \textbf{with a direction}.

\subsubsection{Examples}
The following are various ways to represent a vector going up and to the left equal amounts, with magnitude 1:

\begin{align} % requires amsmath; align* for no eq. number
   \vec{v} &= \vec{AB} & \text{Where } A = (0,0) \text{, and } B = (\frac{\sqrt{2}}{2},\frac{\sqrt{2}}{2}) \\
               &= \left<\frac{\sqrt{2}}{2}\hat{i}, \frac{\sqrt{2}}{2}\hat{j} \right>  & \text{Where } \hat{i} \text{ and } \hat{j} \text{ are unit vectors along the axies}\\
               &= \left<\frac{\sqrt{2}}{2}, \frac{\sqrt{2}}{2}\right> & \text{Similar to above, but with implied } \hat{i} \text{ and } \hat{j}\\
               &= \left(1, \frac{\pi}{4}\right) & \text{In polar coordinates}
\end{align}


\subsection{Vector Magnitude}
As previously stated, a vector is a scalar with direction. But what about times when you want only the scalar? This is called the \emph{magnitude} of a vector, and can be represented as $\left| \vec{v} \right|$ for a vector $\vec{v}$. It is computed by taking the square root of the sum of the squares of each component of a given vector. If this sounds similar to the Pythagorean Theorem, that's because they're actually one in the same: the Pythagorean Theorem is a special case where the vectors are of order 2. In equation form, the magnitude of a vector with order $n$ is:
\begin{align} % requires amsmath; align* for no eq. number
   \left|\vec{v}\right| &= \sqrt{\sum \limits_{i=1} ^n \vec{v}_i^{\,2}}
\end{align}

\subsubsection{Example}
\begin{align} % requires amsmath; align* for no eq. number
   \vec{v} &= \left<2, 4, 9\right> \\ \\
   \left|\vec{v}\right| &= \sqrt{2^2 + 4^2 + 9^2} \\
   &= \sqrt{4 + 16 + 81} \\
   &= \sqrt{101}
\end{align}

\subsection{The Dot Product}
The dot product of two vectors can be thought of as a scalar measure of the extent to which they are in the same direction. From this description, along with some rudimentary trigonometric knowledge, a definition for the dot product arises:

\begin{align} % requires amsmath; align* for no eq. number
   \vec{m} \cdot \vec{v} = \left| \vec{m}\right| \left| \vec{v}\right| \cos\left(\theta\right)
\end{align}

Additionally, the dot product can be computed by multiplying corresponding terms of vectors of the same order together, then summing the results. In a more mathematical notation, the dot product of two order-$n$ vectors, $\vec{m}$ and $\vec{v}$, is:

\begin{align} % requires amsmath; align* for no eq. number
   \vec{m} \cdot \vec{v} = \sum \limits_{i=1} ^n \vec{m}_i \vec{v}_i 
\end{align}

It can be useful to note that when taking the dot product of a vector with itself, the $\cos \left( \theta \right)$ of Equation 1.11 goes to 1, leaving
\begin{align} % requires amsmath; align* for no eq. number
   \vec{v} \cdot \vec{v} = \left| \vec{v} \right|^2
\end{align}\
Important to note: for any unit vector (represented with a hat, '$\hat{a}$'), its cross with itself will always be its magnitude squared, or 1. 
 
 \subsubsection{Usage and Examples}
The dot product sees many uses in physics, where it can be important to know how much a force for instance is acting in a given direction. Additionally, through Equation 1.11, the dot product can be used to calculate angles between arbitrary vectors. Some sample problems follow:

\begin{enumerate}
\item Find the cosine of the angle between the main diagonals of a cube.
\begin{itemize}
\item First, find vectors representing the two diagonals.
For instance:
\begin{align*} % requires amsmath; align* for no eq. number
   \vec{v} &= \left<1,1,1\right> & \vec{m} &= \left<-1,1,1\right>
\end{align*}
\item Combining Equations 1.11 and 1.12, determine an equation for $\theta$.
\begin{align*} % requires amsmath; align* for no eq. number
   \left| \vec{m}\right| \left| \vec{v}\right| \cos\left(\theta\right) &= \sum \limits_{i=1} ^n \vec{m}_i \vec{v}_i  \\ 
   \sqrt{3} \sqrt{3} \cos\left(\theta\right) &= -1 \cdot 1 + 1 \cdot 1 + 1 \cdot 1 \\
   \cos\left(\theta\right) &= \frac{1}{3} 
\end{align*}
\end{itemize}
\item Find the component of $\vec{v} = \left<1,-2,4\right>$ along a $\vec{m} = \left<4,2,-3\right>$ direction.
\begin{itemize}
\item First, find the amount to which they are in the same direction as each other, using a dot product.
\begin{align*}
\left<1,-2,4\right> \cdot \left<4,2,-3\right> &= -12
\end{align*}
\item This however is proportional to $\vec{m}$, so in order to remove the relation, one must simply divide by the magnitude of $\vec{m}$.
\begin{align*}
 \frac{-12}{\left|\vec{m}\right|}  &= \frac{-12}{\sqrt{29}}
\end{align*}
\item Hence, $\vec{v}$ is $\frac{12}{\sqrt{29}}$ units along $\vec{m}$, in a direction opposite to that of $\vec{m}$
\item This concept is called a \textbf{\emph{projection}}, and is commonly used in math and physics. 
\end{itemize}
\end{enumerate}

\subsection{The Cross Product}
While the dot product measures how much two vectors are in the same direction, the cross product can be thought of as a measure of the extent to which two vectors are in different directions. Once more, this concept, along with some trigonometry, gives a definition for the cross product:
\begin{align} % requires amsmath; align* for no eq. number
   \vec{m} \times \vec{v} &= \left| \vec{m}\right| \left| \vec{v}\right| \sin\left(\theta\right)  & \text{\emph{In a direction \textbf{perpendicular} to the two vectors.}}
\end{align}
This direction component is actually very important in some scenarios. For example, in physics, cross products are uses for concepts such as torque, where the direction of $\vec{r} \times \vec{F}$ yields a torsion normal to the plane of a given circle. 
\bigskip

Additionally, the cross product can be computed as the \emph{determinant} of the two vectors, represented as a matrix. Deriving a determinant will be covered in more detail in Section 2, but for now a simple formulas for crossing order-2 vectors is as follows:
\begin{align*} % requires amsmath; align* for no eq. number
   \vec{m} &= \left<m_{i},m_{j}\right> & \vec{v} &= \left<v_{i},v_{j}\right>\\
    \vec{r} &= \left<m_{i},m_{j},m_{k}\right> & \vec{s} &= \left<v_{i},v_{j},v_{k}\right>
   \end{align*}
\begin{align} % requires amsmath; align* for no eq. number
     \vec{m} \times \vec{v} &= m_{i}v_{j} - m_{j}v_{i} \\
    \vec{r} \times \vec{s} &= \left(r_{j}s_{k} - r_{k}s_{j} \right)\hat{i} - \left(r_{i}s_{k} - r_{k}s_{i} \right)\hat{j} + \left(r_{i}s_{j} - r_{j}s_{i} \right)\hat{k}
\end{align}
Do not be overly concerned with memorizing this formula. Just know that it is the determinant of the matrix:
   \[ \begin{pmatrix} % or pmatrix or bmatrix or Bmatrix or ...
      \hat{i} & \hat{j} & \hat{k} \\
      r_{i} & r_{j} & r_{k} \\
      s_{i} &  s_{j} &  s_{k} 
 \end{pmatrix}\]

\bigskip

It can be useful to note that when taking the dot product of a vector with itself, the $\sin \left( \theta \right)$ of Equation 1.14 goes to 0, leaving:
\begin{align} % requires amsmath; align* for no eq. number
   \vec{v} \times \vec{v} = 0
\end{align}


 
 \subsubsection{Usage and Examples}
As previously mentioned, the cross product sees many uses in physics, especially with things involving torque, E\&M, or any other right hand rule oriented field.

Interestingly, the cross product of $n$ order-$n$ vectors can be used to determine the area/volume/etc. of the parallelogram type object formed by joining all those vectors together at one point. To illustrate, sample problems follow:

\begin{enumerate}
\item Derive a formula for the area of a cube with side length $s$.
\begin{itemize}
\item First, find vectors representing the three sides.
For instance:
\begin{align*} % requires amsmath; align* for no eq. number
   \vec{v_1} &= \left<0,0,s\right> & \vec{v_2} &= \left<0,s,0\right> & \vec{v_3} &= \left<s,0,0\right>
\end{align*}
\item Next, cross two of the vectors to determine the area of one face of the solid. 
\begin{align*} % requires amsmath; align* for no eq. number
   \vec{v_1} \times \vec{v_2}  &= \left(v_{1j}v_{2k} - v_{1k}v_{2j} \right)\hat{i} - \left(v_{1i}v_{2k} - v_{1k}v_{2i} \right)\hat{j} + \left(v_{1i}v_{2j} - v_{1j}v_{2i} \right)\hat{k} \\
                                               &= \left(s^2\right)\hat{i} - 0\hat{j} + 0\hat{k}
                                               &= \left(s^2\right)\hat{i}
\end{align*}
\item Now, with the area of one face equal to $s^2$, we must go back to the dot product to determine the extent to which this vector normal to this face lies in direction with the third vector coming up out of the face
\begin{align*} % requires amsmath; align* for no eq. number
   V  &= (s^2)\hat{i} \cdot \vec{v_3} \\
                                               &= (s^2)\hat{i} * (s)\hat{i} + 0 + 0 \\
                                               &= s^3
\end{align*}
\item Thus, we show the volume of the cube to be $s^3$. Not earth shattering, but a worthwhile proof of concept. 
\end{itemize}
\item Find a unit vector perpendicular to both $\vec{v} = \left<1,-2,4\right>$ and $\vec{m} = \left<4,2,-3\right>$.
\begin{itemize}
\item First, find cross product, which will be in a direction perpendicular to both $\vec{v}$ and $\vec{m}$ by definition.
\begin{align*}
\left<1,-2,4\right> \times \left<4,2,-3\right> &= \-2\hat{i} + 19\hat{j} + 10\hat{k}
\end{align*}
\item While this vector is in the proper direction, it is not of proper magnitude, namely 1, so we must then devide this vectors components by its magnitude.
\begin{align*}
 \hat{v}  &= \frac{\-2\hat{i} + 19\hat{j} + 10\hat{k}}{\left| \vec{v} \right|} \\
             &= \frac{-2\hat{i}}{\sqrt{465}} + \frac{19\hat{j}}{\sqrt{465}} + \frac{10\hat{k}}{\sqrt{465}}
\end{align*}
\end{itemize}
\end{enumerate}


\section{Matricies}
Where a vector is a one dimensional list of values, a matrix is a two dimensional list, or an \emph{array}. 

%%%%%%%%%%%%%%%%%%%%%%%%%%%%%%%%%%%%%%%%%%%%%%%%%%%%%%%
% Sample table                                        %
% Source: www1.maths.leeds.ac.uk/latex/TableHelp1.pdf %
%%%%%%%%%%%%%%%%%%%%%%%%%%%%%%%%%%%%%%%%%%%%%%%%%%%%%%%
\begin{table}[ht]
\caption{Sample table} % title of Table
\centering % used for centering table
\begin{tabular}{c c c c}
% centered columns (4 columns)
\hline\hline %inserts double horizontal lines
S. No. & Column\#1 & Column\#2 & Column\#3 \\ [0.5ex]
% inserts table
%heading
\hline % inserts single horizontal line
1 & 50 & 837 & 970 \\
2 & 47 & 877 & 230 \\
3 & 31 & 25 & 415 \\
4 & 35 & 144 & 2356 \\
5 & 45 & 300 & 556 \\ [1ex] % [1ex] adds vertical space
\hline %inserts single line
\end{tabular}
\label{table:nonlin} % is used to refer this table in the text
\end{table}

Duis aute irure dolor in reprehenderit in voluptate velit esse cillum dolore eu fugiat nulla pariatur. Excepteur sint occaecat cupidatat non proident, sunt in culpa qui officia deserunt mollit anim id est laborum. \\ Lorem ipsum list:
\begin{itemize}
\item Mauris sit amet nulla mi, vitae rutrum ante.
\item Maecenas quis nulla risus, vel tincidunt ligula.
\item Nullam ac enim neque, non \emph{dapibus} mauris.
\end{itemize}

\noindent Lorem ipsum dolor sit amet, consectetur adipiscing elit. Duis risus ante, auctor et pulvinar non, posuere ac lacus. Praesent egestas nisi id metus rhoncus ac lobortis sem hendrerit. Etiam et sapien eget lectus interdum posuere sit amet ac urna\footnote{Lorem ipsum dolor sit amet, consectetur adipiscing elit. Duis risus ante, auctor et pulvinar non, posuere ac lacus.}:

\subsection{Matrix Addition}
Lorem ipsum dolor sit amet, consectetur adipiscing elit. Duis risus ante, auctor et pulvinar non, posuere ac lacus. Praesent egestas nisi id metus rhoncus ac lobortis sem hendrerit. Etiam et sapien eget lectus interdum posuere sit amet ac urna. Aliquam pellentesque imperdiet erat, eget consectetur felis malesuada quis. Pellentesque sollicitudin, odio sed dapibus eleifend, magna sem luctus turpis, id aliquam felis dolor eu diam. Etiam ullamcorper, nunc a accumsan adipiscing, turpis odio bibendum erat, id convallis magna eros nec metus. Sed vel ligula justo, sit amet vestibulum dolor. Sed vitae augue sit amet magna ullamcorper suscipit. Quisque dictum ipsum a sapien egestas facilisis.

\subsection{Matrix Multiplication}
Lorem ipsum dolor sit amet, consectetur adipiscing elit. Duis risus ante, auctor et pulvinar non, posuere ac lacus. Praesent egestas nisi id metus rhoncus ac lobortis sem hendrerit. Etiam et sapien eget lectus interdum posuere sit amet ac urna. Aliquam pellentesque imperdiet erat, eget consectetur felis malesuada quis. Pellentesque sollicitudin, odio sed dapibus eleifend, magna sem luctus turpis, id aliquam felis dolor eu diam.

\subsection{Determinants}
Lorem ipsum dolor sit amet, consectetur adipiscing elit. Duis risus ante, auctor et pulvinar non, posuere ac lacus. Praesent egestas nisi id metus rhoncus ac lobortis sem hendrerit. Etiam et sapien eget lectus interdum posuere sit amet ac urna. Aliquam pellentesque imperdiet erat, eget consectetur felis malesuada quis. Pellentesque sollicitudin, odio sed dapibus eleifend, magna sem luctus turpis, id aliquam felis dolor eu diam.

\subsection{Inverting a Matrix}
Lorem ipsum dolor sit amet, consectetur adipiscing elit. Duis risus ante, auctor et pulvinar non, posuere ac lacus. Praesent egestas nisi id metus rhoncus ac lobortis sem hendrerit. Etiam et sapien eget lectus interdum posuere sit amet ac urna. Aliquam pellentesque imperdiet erat, eget consectetur felis malesuada quis. Pellentesque sollicitudin, odio sed dapibus eleifend, magna sem luctus turpis, id aliquam felis dolor eu diam.

\subsection{Linear Systems}
A small set of linear equations is trivial to solve such a system using algebraic techniques such as distribution. However, as the matrices get more complicated, Linear Systems provide a more systematic approach. 
\bigskip

The method relies on the following equation:
\begin{align*}
\vec{A}\vec{x} = \vec{B}
\end{align*}

Where $\vec{A}$ is the matrix of coefficients for a given system, $\vec{x}$ is the unknown array for a solution to the given system,and $\vec{B}$ is the right hand side of a given system. In example: given
\begin{align*}
ax + by &= m\\
cx + dy &= n
\end{align*}
then,
\begin{align*}
\vec{A} &=    
\begin{pmatrix} % or pmatrix or bmatrix or Bmatrix or ...
      a & b \\
      c & d \\
 \end{pmatrix} &
 \vec{x} &=    
\begin{pmatrix} % or pmatrix or bmatrix or Bmatrix or ...
       x \\
       y \\
 \end{pmatrix} &
  \vec{B} &=    
\begin{pmatrix} % or pmatrix or bmatrix or Bmatrix or ...
       m \\
       n \\
 \end{pmatrix}
 \end{align*}


\end{document}
