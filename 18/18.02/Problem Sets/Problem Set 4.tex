%%%%%%%%%%%%%%%%%%%%%%%%%%%%%%%%%%%%%%%%%%%%%%%%%%%
%% LaTeX book template                           %%
%% Author:  Amber Jain (http://amberj.devio.us/) %%
%% License: ISC license                          %%
%%%%%%%%%%%%%%%%%%%%%%%%%%%%%%%%%%%%%%%%%%%%%%%%%%%

\documentclass[a4paper,11pt,oneside]{book}

\usepackage[T1]{fontenc}
\usepackage[utf8]{inputenc}
\usepackage{lmodern}
\usepackage{csquotes}
\usepackage{amsmath}

\usepackage{breqn}



\usepackage[margin=1.25in]{geometry}

%%%%%%%%%%%%%%%%%%%%%%%%%%%%%%%%%%%%%%%%%%%%%%%%%%%%%%%%%
% Source: http://en.wikibooks.org/wiki/LaTeX/Hyperlinks %
%%%%%%%%%%%%%%%%%%%%%%%%%%%%%%%%%%%%%%%%%%%%%%%%%%%%%%%%%
\usepackage{hyperref}
\usepackage{graphicx}
\usepackage[english]{babel}

%%%%%%%%%%%%%%%%%%%%%%%%%%%%%%%%%%%%%%%%%%%%%%%%%%%%%%%%%%%%%%%%%%%%%%%%%%%%%%%%
% 'dedication' environment: To add a dedication paragraph at the start of book %
% Source: http://www.tug.org/pipermail/texhax/2010-June/015184.html            %
%%%%%%%%%%%%%%%%%%%%%%%%%%%%%%%%%%%%%%%%%%%%%%%%%%%%%%%%%%%%%%%%%%%%%%%%%%%%%%%%
\newenvironment{dedication}
{
   \cleardoublepage
   \thispagestyle{empty}
   \vspace*{\stretch{1}}
   \hfill\begin{minipage}[t]{0.66\textwidth}
   \raggedright
}
{
   \end{minipage}
   \vspace*{\stretch{3}}
   \clearpage
}

%%%%%%%%%%%%%%%%%%%%%%%%%%%%%%%%%%%%%%%%%%%%%%%%
% Chapter quote at the start of chapter        %
% Source: http://tex.stackexchange.com/a/53380 %
%%%%%%%%%%%%%%%%%%%%%%%%%%%%%%%%%%%%%%%%%%%%%%%%
\makeatletter
\renewcommand{\@chapapp}{}% Not necessary...
\newenvironment{chapquote}[2][2em]
  {\setlength{\@tempdima}{#1}%
   \def\chapquote@author{#2}%
   \parshape 1 \@tempdima \dimexpr\textwidth-2\@tempdima\relax%
   \itshape}
  {\par\normalfont\hfill--\ \chapquote@author\hspace*{\@tempdima}\par\bigskip}
\makeatother

%%%%%%%%%%%%%%%%%%%%%%%%%%%%%%%%%%%%%%%%%%%%%%%%%%%
% First page of book which contains 'stuff' like: %
%  - Book title, subtitle                         %
%  - Book author name                             %
%%%%%%%%%%%%%%%%%%%%%%%%%%%%%%%%%%%%%%%%%%%%%%%%%%%

% Book's title and subtitle
\title{\Huge \textbf{18.02: Multivariable Calculus} \\ \huge Problem Set 2 \\ \large As taught by \textsc{Prof. John Bush}, MIT}
% Author
\author{\textsc{Jackson Kearl}}


\begin{document}

\maketitle

%%%%%%%%%%%%%%%%%%%%%%%%%%%%%%%%%%%%%%%%%%%%%%%%%%%%%%%%%%%%%%%%%%%%%%%%
% Auto-generated table of contents, list of figures and list of tables %
%%%%%%%%%%%%%%%%%%%%%%%%%%%%%%%%%%%%%%%%%%%%%%%%%%%%%%%%%%%%%%%%%%%%%%%%
\let\cleardoublepage\clearpage
\tableofcontents
\let\cleardoublepage\clearpage
\mainmatter

%%%%%%%%%%%
% Preface %
%%%%%%%%%%%
\chapter*{Introduction}

\begin{chapquote}{Lewis Carroll, \textit{Alice in Wonderland}}
``Begin at the beginning,'' the King said, gravely, ``and go on till you
come to an end; then stop.''
\end{chapquote}

\section*{Course Content}
This course covers aspects of calculus involving:
\begin{displayquote}
Calculus of several variables. Vector algebra in 3-space, determinants, matrices.
 Vector-valued functions of one variable, space motion. Scalar functions of several variables: partial differentiation, gradient, optimization techniques.  Double integrals and line integrals in the plane; exact differentials and  conservative fields; Green's theorem and applications, triple integrals, line and surface integrals in space, Divergence theorem, Stokes' theorem;  applications.\footnote{\url{http://student.mit.edu/catalog/m18a.html##18.02}}
\end{displayquote}


\section*{Structure of Homework}
% You might want to add short description about each chapter in this book.
Problems will be arranged as they are on the assigment page. Problems broken into parts one and two, as per assignment page. Within part one, problems divided first by Lecture number, then by book section.

\section*{Distribution}
All relevant notes and information are made available on github\footnote{\url{https://github.com/JacksonKearl/MIT}}.
This includes:
\begin{itemize}
  \item Plaintext notes from lecture and recitation
  \item The raw .tex and .md source for these PDF's
  \item These PDF files
\end{itemize}

\noindent Licensed under Creative Commons Attribution-ShareAlike 4.0 International License.\footnote{\url{http://creativecommons.org/licenses/by-sa/4.0/}}
\let\cleardoublepage\clearpage


%%%%%%%%%%%%%%%%
% NEW CHAPTER! %
%%%%%%%%%%%%%%%%
\begingroup
\let\cleardoublepage\clearpage
\part{}
\endgroup


\chapter{Lecture 10}

\section{Section 10.9}

\subsection{Problem 2}\label{problem-2}

\begin{enumerate}
\item
  Define Functions:

  \begin{align*}
  \text{Function: }& &f(x,y) &= x + y \\
  \text{Constraint: }& &x^2 + 4y^2 &= 1
  \end{align*}


\item
  Rearrange such that the condition is satisfied when \(g(x,y) = 0\):
  \[g(x,y) = x^2 + 4y^2 - 1\]
\item
  Plug into Lagrange Multiplier equation,
  \(L(x,y,\lambda) = f(x,y)-\lambda g(x,y)\):
  \[L(x, y, \lambda) = x + y - \lambda(x^2 + 4y^2 - 1)\]
\item
  Take partials with respect to \(x\), \(y\), and \(\lambda\), and set
  to \(0\):
  \[\frac{\partial L}{\partial x} = 1 - 2\lambda x = 0 \Longrightarrow x = \frac{1}{2\lambda}\]
  \[\frac{\partial L}{\partial y} = 1 - 8\lambda y = 0 \Longrightarrow y = \frac{1}{8\lambda}\]
  \[\frac{\partial L}{\partial \lambda} = x^2 + 4y^2 - 1 = 0\]
\item
  Solve as a system of equations:

  \begin{align*}
  \frac{1}{2\lambda}^2 + 4 \frac{1}{8\lambda}^2 -1 &= 0 \\
  \frac{1}{4\lambda^2} + 4 \frac{1}{64\lambda^2} -1 &= 0 \\
  \frac{4}{16\lambda^2} + \frac{1}{16\lambda^2} -1 &= 0 \\
  \frac{5}{16\lambda^2} &= 1 \\
  \lambda &= \pm \frac{\sqrt{5}}{4}
  \end{align*}


\item
  Substitute back to find \(x\) and \(y\):\\


  \begin{align*}
  x = \frac{\pm4}{2\sqrt{5}} &= \frac{\pm2}{\sqrt{5}}\\
  y = \frac{\pm4}{8\sqrt{5}} &= \frac{\pm1}{2\sqrt{5}}
  \end{align*}


\item
  Empirically determine minimum and maximum from original \(f(x,y)\):


  \begin{align*}
  \text{Minimum: }& &f \left(\frac{-2}{\sqrt{5}}, \frac{-1}{2\sqrt{5}}\right) = \frac{-2}{\sqrt{5}} + \frac{-1}{2\sqrt{5}} &= \frac{-5}{2\sqrt{5}}\\
  \text{Maximum: }& &f \left(\frac{2}{\sqrt{5}}, \frac{1}{2\sqrt{5}}\right) = \frac{2}{\sqrt{5}} +\frac{1}{2\sqrt{5}} &= \frac{5}{2\sqrt{5}}
  \end{align*}


\end{enumerate}

\subsection{Problem 8}\label{problem-8}

\begin{enumerate}

\item
  Define Functions:

  \begin{align*}
  \text{Function: }& &f(x,y,z) &= 3x + 2y + z\\
  \text{Constraint: }& &x^2 + y^2 + z^2 &= 1
  \end{align*}


\item
  Rearrange such that the condition is satisfied when \(g(x,y,z) = 0\):
  \[g(x,y) = x^2 + y^2 + z^2 - 1\]
\item
  Plug into Lagrange Multiplier equation,
  \(L(x,y,z,\lambda) = f(x,y,z)-\lambda g(x,y,z)\)
  \[L(x, y, z, \lambda) = 3x + 2y + z - \lambda(x^2 + y^2 + z^2 - 1)\]
\item
  Take partials with respect to \(x\), \(y\), \(z\), and \(\lambda\),
  and set to \(0\):
  \[\frac{\partial L}{\partial x} = 3 - 2\lambda x = 0 \Longrightarrow x = \frac{3}{2\lambda}\]
  \[\frac{\partial L}{\partial y} = 2 - 2\lambda y = 0 \Longrightarrow y = \frac{1}{\lambda}\]
  \[\frac{\partial L}{\partial z} = 1 - 2\lambda z = 0 \Longrightarrow z = \frac{1}{2\lambda}\]
  \[\frac{\partial L}{\partial \lambda} = x^2 + y^2 + z^2 - 1 = 0\]
\item
  Solve as a system of equations:

  \begin{align*}
  \frac{3}{2\lambda}^2 + \frac{1}{\lambda}^2 + \frac{1}{2\lambda}^2-1 &= 0 \\
  \frac{9}{4\lambda^2} + \frac{1}{\lambda^2} + \frac{1}{4\lambda^2} -1 &= 0 \\
  \frac{9}{4\lambda^2} + \frac{4}{4\lambda^2} + \frac{1}{4\lambda^2} -1 &= 0 \\
  \frac{14}{4\lambda^2} -1 &= 0 \\
  \frac{7}{2\lambda^2} &= 1 \\
  \lambda &= \pm \frac{\sqrt{14}}{2}
  \end{align*}


\item
  Substitute back to find \(x\) and \(y\):

  \begin{align*}
  x &= \pm\frac{3}{\sqrt{14}}\\
  y &= \pm\frac{2}{\sqrt{14}}\\
  z &= \pm\frac{1}{\sqrt{14}}
  \end{align*}


\item
  Empirically determine min and max from original \(f(x,y)\):

  \begin{align*}
  \text{Min: }& &f \left(\frac{-3}{\sqrt{14}}, \frac{-2}{\sqrt{14}}, \frac{-1}{\sqrt{14}}\right) =\frac{-3}{\sqrt{14}}+ \frac{-2}{\sqrt{14}}+ \frac{-1}{\sqrt{14}} = \frac{-6}{\sqrt{14}}\\
  \text{Max: }& &f \left(\frac{3}{\sqrt{14}}, \frac{2}{\sqrt{14}}, \frac{1}{\sqrt{14}}\right) =\frac{3}{\sqrt{14}}+ \frac{2}{\sqrt{14}}+ \frac{1}{\sqrt{14}} = \frac{6}{\sqrt{14}}
  \end{align*}


\end{enumerate}

\subsection{Problem 38}\label{problem-38}

\begin{enumerate}

\item
  Define Functions:

  \begin{align*}
  \text{Function: }& &f(x,y) &= x^2 + y^2\\
  \text{Constraint: }& &x^2 + xy + y^2 &= 3
  \end{align*}


\item
  Rearrange such that the condition is satisfied when \(g(x,y) = 0\):
  \[g(x,y) = x^2 + xy + y^2 - 3\]
\item
  Plug into Lagrange Multiplier equation,
  \(L(x,y,\lambda) = f(x,y)-\lambda g(x,y)\)
  \[L(x, y, \lambda) = x^2 + y^2 - \lambda(x^2 + xy + y^2 - 3)\]
\item
  Take partials with respect to \(x\), \(y\), and \(\lambda\), and set
  to \(0\):
  \[\frac{\partial L}{\partial x} = 2x - 2\lambda x - \lambda y= 0 \Longrightarrow x = \frac{\lambda y}{2 - 2\lambda}\]
  \[\frac{\partial L}{\partial y} = 2y - 2\lambda y - \lambda x= 0 \Longrightarrow y = \frac{\lambda x}{2 - 2\lambda}\]
  \[\frac{\partial L}{\partial \lambda} = x^2 + xy + y^2 - 3 = 0\]
\item
  Solve as a system of equations:

  \begin{align*}
  x &= \frac{\lambda y}{2-2\lambda}& && y &= \frac{\lambda x}{2-2\lambda}\\
  \frac{x}{y} &= \frac{\lambda}{2-2\lambda}& && \frac{y}{x} &= \frac{\lambda}{2-2\lambda}\\
  &&\frac{x}{y} &= \frac{y}{x}\\
  &&x^2 &= y^2\\
  &&y &= \pm x\\
  g(x,y) = 3x^2 - 3 &= 0 &&& g(x,y) =x^2-3 &= 0\\
  3x^2 &= 3 &&& x^2 &= 3\\
  x^2 &= 1 &&& x^2 &= 3\\
  x =y &= \pm1 &&& x =-y&= \pm\sqrt{3}\\
  \end{align*}


\item
  Substitute back to find \(x\) and \(y\):

  \begin{align*}
  (x,y) = (1,1),\space(-1,-1),\space(\sqrt{3},-\sqrt{3}),\space(-\sqrt{3},\sqrt{3})
  \end{align*}


\item
  Empirically determine min and max from original \(f(x,y)\):

  \begin{align*}
  \text{Distance from Origin:} &&D(x,y) &= \sqrt{x^2 + y^2}\\
  \text{Minimums @ }(1,1),\space(-1,-1): & &D(1,1) =  D(-1,-1)&= \sqrt{2}\\
  \text{Maximums @ }(-\sqrt{3},\sqrt{3}),\space(\sqrt{3},-\sqrt{3}): & &D(-\sqrt{3}),\sqrt{3})&= \sqrt{6}
  \end{align*}


\end{enumerate}

\subsection{Problem 52}\label{problem-52}

\begin{enumerate}
\def\labelenumi{\arabic{enumi}.}
\item
  Define Functions:

  \begin{align*}
  \text{Function: }& &f(x,y) &= (x-3)^2 + (y-2)^2\\
  \text{Constraint: }& &4x^2 + 9y^2 &= 36
  \end{align*}


\item
  Rearrange such that the condition is satisfied when \(g(x,y) = 0\):
  \[g(x,y) = 4x^2 + 9y^2 -36\]
\item
  Plug into Lagrange Multiplier equation,
  \(L(x,y,\lambda) = f(x,y)-\lambda g(x,y)\)
  \[L(x, y, \lambda) = (x-3)^2 + (y-2)^2 - \lambda(4x^2 + 9y^2 - 36)\]
\item
  Take partials with respect to \(x\), \(y\), and \(\lambda\), and set
  to \(0\):
  \[\frac{\partial L}{\partial x} = 2x - 6 -8\lambda x= 0 \Longrightarrow x = \frac{3}{1 - 4\lambda}\]
  \[\frac{\partial L}{\partial y} = 2y - 4-18\lambda y= 0 \Longrightarrow y = \frac{2}{1 - 9\lambda}\]
  \[\frac{\partial L}{\partial \lambda} = 4x^2 + 9y^2 - 36 = 0\]
\item
  Solve as a system of equations, using CAS software:

  \begin{align*}
  4x^2 + 9y^2 - 36 &= 0\\
  4\left(\frac{3}{1 - 4\lambda}\right)^2 + 9\left(\frac{2}{1 - 9\lambda}\right)^2 - 36 &= 0\\
  \lambda &= [0.5103, -0.0684]
  \end{align*}


\item
  Substitute back to find \(x\) and \(y\):

  \[(x,y) = \left(\frac{3}{1 - 4\lambda},\frac{2}{1 - 9\lambda}\right)\]

  \begin{align*}
  (x,y) &= \left(\frac{3}{1 - 4(0.5103)},\frac{2}{1 - 9(0.5103)}\right) & \text{OR  } & (x,y) = \left(\frac{3}{1 - 4(-0.0684)},\frac{2}{1 - 9(-0.0684)}\right)\\
  (x,y) &= \left(-2.88, -0.557\right) & \text{OR  } & (x,y) = (2.36,1.24)
  \end{align*}


\item
  Empirically determine min and max from original \(f(x,y)\):

  \begin{align*}
  \text{Distance from Origin:} &&D(x,y) = \sqrt{(x-3)^2 + (y-2)^2}\\
  \text{Minimum @ }(2.36,1.24): & &D(2.36,1.24) = 0.993\\
  \text{Maximum @ }(-2.88, -0.557): & &D(-2.88, -0.557)= 6.411
  \end{align*}


\end{enumerate}

\section{Section 13.10}\label{section-13.10}

\subsection{Problem 4}\label{problem-4}

\begin{enumerate}

\item
  Calculate partials, set to 0 to find critical points:

  \begin{align*}
  \frac{\partial f}{\partial x} = y+3= 0 &\Longrightarrow y = -3\\
  \frac{\partial f}{\partial y} = x - 2= 0 &\Longrightarrow x = 2
  \end{align*}


\item
  Evaluate discriminant at critical point, \((2,-3)\):

  \begin{align*}
  \Delta &= f_{xx}(x,y)\space f_{yy}(x,y) - [f_{xy}(x,y)]^2\\
  &= 0\cdot0-1\\
  &= -1
  \end{align*}

   The discriminant at \((2,-3)\) is negative, meaning there is a
  saddle point here, not a minimum or maximum.
\end{enumerate}

\subsection{Problem 10}\label{problem-10}

\begin{enumerate}

\item
  Calculate partials, set to 0 to find critical points:

  \begin{align*}
  \frac{\partial f}{\partial x} = 3y - 3x^2 &= 0 &\Longrightarrow y = x^2\\
  \frac{\partial f}{\partial y} = 3x - 3y^2 &= 0 &\Longrightarrow x = y^2\\
  x &= x^4 &\Longrightarrow (x,y) = (0,0)\\
  1 &= x^3 &\Longrightarrow (x,y) = (1,1)\\
  \end{align*}


\item
  Evaluate discriminant at critical points \((0,0)\), and \((1,1)\):

  \begin{align*}
  \Delta(x,y) &= f_{xx}(x,y)\space f_{yy}(x,y) - [f_{xy}(x,y)]^2\\
  &= -6x \cdot -6y - [3]^2\\
  \Delta(0,0)&=0 \cdot 0 - 9 = -9\\
  \Delta(1,1)&= -6 \cdot -6 - 9 = 27
  \end{align*}

   The discriminant at \((0,0)\) is negative, meaning it is a saddle
  point, not a minima or maxima.

  However, the discriminant at \((1,1)\) is positive, meaning it is
  either a minima or maxima. Because \(f_{xx}(1,1)\) is negative,
  \((1,1)\) is a maxima.
\end{enumerate}

\chapter{Lecture 13}\label{lecture-13}

\section{Section 13.7}\label{section-13.7}

\subsection{Problem 2}\label{problem-2-1}

\begin{enumerate}

\item
  Finding \(\frac{dw}{dt}\) via chain rule:

  \begin{align*}
  \frac{dw}{dt} &= \frac{\partial w}{\partial u} \frac{d u}{dt} + \frac{\partial w}{\partial v} \frac{d v}{dt}\\
  &= \frac{2u}{(u^2+v^2)^2} (-2\sin(2t))+ \frac{2v}{(u^2+v^2)^2} (2\cos(2t))\\
  &= \frac{2\cos(2t)}{(\cos(2t)^2+\sin(2t)^2)^2} (-2\sin(2t))+ \frac{2\sin(2t)}{(\cos(2t)^2+\sin(2t)^2)^2} (2\cos(2t))\\
  &=2\cos(2t)(-2\sin(2t)) + 2\sin(2t) (2\cos(2t)\\
  &=0
  \end{align*}


\item
  Finding \(\frac{dw}{dt}\) via substitution:

  \begin{align*}
  w(u,v) &= \frac{1}{\cos(2t)^2+\sin(2t)^2}\\
  &= \frac{1}{1} = 1\\
  \frac{dw}{dt} &= 0\\
  \end{align*}


\end{enumerate}

\subsection{Problem 4}\label{problem-4-1}

\begin{enumerate}

\item
  Finding \(\frac{dw}{dt}\) via chain rule:

  \begin{align*}
  \frac{dw}{dt} &= \frac{\partial w}{\partial u} \frac{d u}{dt} + \frac{\partial w}{\partial v} \frac{d v}{dt} + \frac{\partial w}{\partial z} \frac{d z}{dt}\\
  &= \frac{1}{u+v+z}[2\sin(t)\cos(t)+(-2)\cos(t)\sin(t)+2t]\\
  &= \frac{2t}{1+t^2}\\
  \end{align*}


\item
  Finding \(\frac{dw}{dt}\) via substitution:

  \begin{align*}
  w(u,v,z) &= \ln(\cos^2 t + \sin^2 t + t^2)\\
  &= \ln(1 + t^2)\\
  \frac{dw}{dt} &= \frac{2t}{1+t^2}\\
  \end{align*}


\end{enumerate}

\subsection{Problem 6}\label{problem-6}

\begin{enumerate}

\item
  Finding \(\frac{\partial w}{\partial t}\):

  \begin{align*}
  \frac{\partial w}{\partial t} &= \frac{\partial w}{\partial p} \frac{d p}{dt} + \frac{\partial w}{\partial q} \frac{d q}{dt} + \frac{\partial w}{\partial r} \frac{d r}{dt}\\
  &= q\sin(r) \cdot 1 + p\sin(r)\cdot -1 + pq\cos(r) \cdot s\\
  &= pqs \cos(r) + q\sin(r) - p\sin(r)\\
  \end{align*}


\item
  Finding \(\frac{\partial w}{\partial s}\):

  \begin{align*}
  \frac{\partial w}{\partial s} &= \frac{\partial w}{\partial p} \frac{d p}{ds} + \frac{\partial w}{\partial q} \frac{d q}{ds} + \frac{\partial w}{\partial r} \frac{d r}{ds}\\
  &= q\sin(r) \cdot 2 + p\sin(r)\cdot 1 + pq\cos(r) \cdot t\\
  &= pqt \cos(r) + 2q\sin(r) + p\sin(r)\\
  \end{align*}


\end{enumerate}

\subsection{Problem 16}\label{problem-16}

\begin{enumerate}

\item
  Finding \(\frac{\partial p}{\partial x}\):

  \begin{align*}
  \frac{\partial p}{\partial x} &= \frac{\partial p}{\partial v} \frac{\partial v}{\partial x} + \frac{\partial p}{\partial u} \frac{\partial u}{\partial x}\\
  \end{align*}


\item
  Finding \(\frac{\partial p}{\partial y}\):

  \begin{align*}
  \frac{\partial p}{\partial y} &= \frac{\partial p}{\partial v} \frac{\partial v}{\partial y} + \frac{\partial p}{\partial u} \frac{\partial u}{\partial y}\\
  \end{align*}


\item
  Finding \(\frac{\partial p}{\partial z}\):

  \begin{align*}
  \frac{\partial p}{\partial x} &= \frac{\partial p}{\partial v} \frac{\partial v}{\partial z} + \frac{\partial p}{\partial u} \frac{\partial u}{\partial z}\\
  \end{align*}


\item
  Finding \(\frac{\partial p}{\partial t}\):

  \begin{align*}
  \frac{\partial p}{\partial y} &= \frac{\partial p}{\partial v} \frac{\partial v}{\partial t} + \frac{\partial p}{\partial u} \frac{\partial u}{\partial t}\\
  \end{align*}


\end{enumerate}

\subsection{Problem 20}\label{problem-20}

\begin{enumerate}

\item
  Rearrange to get \(z = f(x,y,z) + g(x,y,z) + h(x,y,z)\):

  \begin{align*}
  z &= f(x,y,z) + g(x,y,z) + h(x,y,z)\\
  f(x,y,z) &= \frac{x^2}{y}\\
  g(x,y,z) &= \frac{y^2}{x}\\
  h(x,y,z) &= \frac{z^3}{xy}
  \end{align*}


\item
  Use chain rule to find \(\frac{\partial z}{\partial x}\):

  \begin{align*}
  \frac{\partial z}{\partial x} &=
          \frac{\partial z}{\partial f} \frac{\partial f}{\partial x} +
          \frac{\partial z}{\partial g} \frac{\partial g}{\partial x} +
          \frac{\partial z}{\partial g} \frac{\partial g}{\partial x} \\
        &= 1\cdot \frac{2x}{y} + 1 \cdot -\frac{y^2}{x^2} + 1 \cdot -\frac{z^3}{x^2y}\\
        &= \frac{2x}{y}-\frac{y^2}{x^2} -\frac{z^3}{x^2y}\\
        &= \frac{2x^3-y^3-z^3}{x^2y}
  \end{align*}


\item
  Use chain rule to find \(\frac{\partial z}{\partial y}\):

  \begin{align*}
  \frac{\partial z}{\partial y} &=
          \frac{\partial z}{\partial f} \frac{\partial f}{\partial y} +
          \frac{\partial z}{\partial g} \frac{\partial g}{\partial y} +
          \frac{\partial z}{\partial g} \frac{\partial g}{\partial y} \\
        &= 1\cdot -\frac{x^2}{y^2} + 1 \cdot \frac{2y}{x} + 1 \cdot -\frac{z^3}{xy^2}\\
        &= -\frac{x^2}{y^2} + \frac{2y}{x} -\frac{z^3}{xy^2}\\
        &= \frac{2y^3-x^3-z^3}{xy^2}
  \end{align*}


\end{enumerate}

\subsection{Problem 40}\label{problem-40}



\begin{align*}
\left(\frac{\partial w}{\partial x}\right)^2 + \left(\frac{\partial w}{\partial y}\right)^2 &= \left(\frac{\partial w}{\partial r}\right)^2 + \frac{1}{r^2} \left(\frac{\partial w}{\partial \theta}\right)^2\\
&= \left(\frac{\partial w}{\partial x}\frac{\partial x}{\partial r} + \frac{\partial w}{\partial y}\frac{\partial y}{\partial r}\right)^2 + \frac{1}{r^2} \left(\frac{\partial w}{\partial x}\frac{\partial x}{\partial \theta} + \frac{\partial w}{\partial y}\frac{\partial y}{\partial \theta}\right)^2\\
&= \left(\frac{\partial w}{\partial x}\cos(\theta) + \frac{\partial w}{\partial y}\sin(\theta)\right)^2 + \frac{1}{r^2} \left(\frac{\partial w}{\partial x}(-r\sin(\theta)) + \frac{\partial w}{\partial y}r\cos(\theta)\right)^2\\
&= \left(\frac{\partial w}{\partial x}\cos(\theta) + \frac{\partial w}{\partial y}\sin(\theta)\right)^2 + \left(\frac{\partial w}{\partial y}(\cos(\theta)) - \frac{\partial w}{\partial x}\sin(\theta)\right)^2\\
&= \left(\frac{\partial w}{\partial x}\right)^2\cos^2(\theta) +
    2\frac{\partial w}{\partial y}\frac{\partial w}{\partial x}\sin(\theta)\cos(\theta) +
    \left(\frac{\partial w}{\partial y}\right)^2\sin^2(\theta) + \\
    &\hspace{30pt}\left(\frac{\partial w}{\partial y}\right)^2\cos^2(\theta) -
    2\frac{\partial w}{\partial y}\frac{\partial w}{\partial x}\sin(\theta)\cos(\theta) +
    \left(\frac{\partial w}{\partial x}\right)^2\sin^2(\theta)\\
&= \left(\frac{\partial w}{\partial x}\right)^2\cos^2(\theta) +
    \left(\frac{\partial w}{\partial y}\right)^2\sin^2(\theta) +
    \left(\frac{\partial w}{\partial y}\right)^2\cos^2(\theta) +
    \left(\frac{\partial w}{\partial x}\right)^2\sin^2(\theta)\\
&=  \left(\frac{\partial w}{\partial x}\right)^2\left(\cos^2(\theta)+ \sin^2(\theta)\right)+
    \left(\frac{\partial w}{\partial y}\right)^2\left(\sin^2(\theta)+ \cos^2(\theta)\right) \\
&= \left(\frac{\partial w}{\partial x}\right)^2 + \left(\frac{\partial w}{\partial y}\right)^2\\
\end{align*}



\end{document}
